\documentclass[11pt]{article}

\title{Compilation of Pattern Matching}
\usepackage{amsmath }
\usepackage{framed}
\usepackage{proof}
\usepackage{enumerate,xspace,stmaryrd}
\usepackage{amsmath,amssymb,latexsym}
\usepackage[dvips]{graphics}
\usepackage{textcomp,xspace}
\usepackage[top=2cm, bottom=2.5cm, right=2.5cm, left=2.5cm]{geometry}\usepackage{latexsym}  %For plain TeX symbols, such as \Box used in \qed
\usepackage{euscript}  %%Euler Script font
\usepackage {mdframed}
\usepackage{ifpdf}
\usepackage {alltt}
\renewcommand{\ttdefault}{txtt}
\ifpdf
  \usepackage{epstopdf}
\fi 
\usepackage{etoolbox}
\BeforeBeginEnvironment{tabular}{\begin{center}\small}
\AfterEndEnvironment{tabular}{\end{center}}

\mdfdefinestyle{MyFrame}{%
    linecolor=black,
    linewidth=1pt
    innertopmargin=\baselineskip,
    innerbottommargin=\baselineskip,
    leftmargin  = 2cm
    rightmargin = 2cm,
    frametitlealignment = \center
    }

\mdfdefinestyle{MyFrameSp}{%
    linecolor=black,
    linewidth=1pt
    innertopmargin=\baselineskip,
    innerbottommargin=\baselineskip,
    leftmargin  = .5cm
    rightmargin = .5cm,
    frametitlealignment = \center
    }

\begin{document}

\maketitle
\section {Compilation of MPL Programs to Core MPL}

Once an MPL program is processed by the frontend, i.e the lexer and the parser, an AST is generated which is a faithful representation of the original program. This AST is used to type infer the MPL programs.
~~\\~~\\ 
The next step in compilation of MPL programs is to compile {\em pattern-matching} to the {\em case} statements and {\em lambda lifting} the MPL programs which involves making all the local functions global. {\em Core MPL(CMPL)} is an intermediate language of MPL which gets rid of the syntactic sugar from MPL programs and represents every program in terms of a reduced set of language constructs.

\section {Example of Pattern-Matching Compilation} \label {ExPattMatch}
{\em Pattern-matching} is a syntactic feature of MPL which allows the programmer to write readable programs such as the following :- 
\begin{alltt}
fun myZip = 
     [],   []   -> [] 
     [],   ys   -> []
     xs,   []   -> []
     x:xs, y:ys -> x:append (xs,ys) 
\end{alltt}
In the above example, the constructors on the left hand side of the fucntion 
defintion ({\em Nil and Cons}) are called {\em patterns}. In the CMPL stage, the patterns are compiled into {\em case} statements. For example, CMPL representation of the {\em myZip} function will look like the following.
\begin{alltt}
fun myZip =
    xl,yl -> 
        case xl of
          [] -> 
            case yl of  
               []   -> []
               y:ys -> [] 
          (x:xs) -> 
            case yl of 
              []   -> []
              y:ys -> <x,y>:myZip (xs,ys)

\end{alltt}

\section {Steps of Pattern-Matching Compilation}
The compilation from {\em pattern-matching} to the {\em case} statement  can be done in following steps :-
\begin {enumerate}
   \item Take a function that uses {\em pattern-matching} and represent it with  the {\em Match} construct. The Match construct is described in section \ref {MatchCon}.
   \item Reduce the {\em Match} construct repeatedly until the base case is reached and reduction is not possible anymore. The reduction of the {\em Match} construct has been described in the section \ref {RedMatch}. 
   \item Once the base case is reached, the {\em pattern-matching} is successfully compiled to case statement. Handling the base case has been described in 
   section \ref {BaseCase}
\end {enumerate}

\subsection {Match Construct} \label{MatchCon} 
{\em Match} construct can be described as the following Haskell data type :- 

\begin{alltt}
data Match    = Match [String] [Equation] Term 

type Equation = ([Pattern],Term)
\end{alltt}
The first argument of the {\em Match} constructor is a list of variable names. The number of elements in this list is the same as the number of patterns in any given line of {\em pattern-matching}. It should be noted that every well formed function definition that uses {\em pattern-matching} should have same number of patterns in every {\em pattern-matching} line. 
The second argument of {\em Match} is an equation list. Every equation in the list captures a line of {\em pattern-matching}. 
The last argument is a term acts like a default term while dealing with missing constructors of a data type in {\em pattern-matching}.
~~\\~~\\ 
An example of a function represented with a {\em Match} construct is as below. 
\begin{alltt}
fun myFun =
    a,[],     ys     -> P a ys  
    a,(x:xs), []     -> Q a xs 
    a,(x:xs), (y:ys) -> R x xs y ys 
\end{alltt}
An AST representation of the above function can be following :- 
\begin{alltt}
    Fun ``myFun''
            [
             ([
               VarPatt ``a'',
               ConsPatt (``Nil'',[]),
               VarPatt ``ys''
              ],
                    TCons (``P'',[TVar ``a'',TVar ``ys''])),
             ([
               VarPatt ``a'',
               ConsPatt (``Cons'',[VarPatt ``x'',VarPatt ``xs'']),
               ConsPatt (``Nil'',[])
               ],
                    TCons (``Q'',[TVar ``a'',TVar ``xs''])),
             ([
               VarPatt ``a'',
               ConsPatt (``Cons'',[VarPatt ``x'',VarPatt ``xs'']),
               ConsPatt (``Cons'',[VarPatt ``y'',VarPatt ``ys''])
              ],
                    TCons (``R'',[TVar ``x'',TVar ``xs'',TVar ``y'',TVar ``ys''])
             )
            ]           
\end{alltt}
{\em VarPatt} is used to represent variable patterns. It takes an argument of type String, which is the variable that it is representing.
{\em ConsPatt} is used to represent constructor patterns. This takes as argument a pair. First element of the pair is the constructor name and second element is the list of arguments that this constructor takes as input. For example -
\begin{alltt}
ConsPatt (``Cons'',[VarPatt ``x'',VarPatt ``xs''])
\end{alltt}
Three constructors P, Q and R have been used as terms in the definition of {\em myFun}. 
~~\\~~\\ 
Like {\em ConsPatt} represents constructor patterns, {\em TCons}
represents constructor terms. {\em TCons} takes a pair as argument. The first element of the pair is the name of the constructor and second element is the list of terms that this constructor takes as input.
~~\\~~\\
For example - 
\begin{alltt}
TCons (``S'',[TVar ``x'',TVar ``xs'',TVar ``y'',TVar ``ys''])
\end{alltt}
This represents a term constructor {\em S} taking four terms as inputs.
~~\\~~\\
Given an AST for the function, the corresponding {\em Match} representation of the function can be easily generated. The {\em Match} representation for the body of {\em myFun} function is given below. Note that instead of representing {\em Match} with the heavy data type representation like the AST, a more readable and succinct representation has been used.  

\begin{alltt}
fnMatch = Match [u\(\sb{1}\),u\(\sb{2}\),u\(\sb{3}\)]
            [
              ([a,Nil,ys],               P a ys),
              ([a,Cons(x,xs),Nil],       Q a xs),
              ([a,Cons(x,xs),Cons(y,ys)],R x xs y ys)
            ]
            t\(\sp{\prime}\) 
\end{alltt}
~~\\
In the {\em Match} construct above $t^\prime$ is the default term to be used with constructors of a data type that are missing from the {\em pattern-matching}. It can be an error message or an exception. The final function definition for {\em myFun} with compiled pattern matching can be obtained as follows.
\begin{alltt}
fun myFun =
    u1,u2,u3 -> reduced fmMatch
\end{alltt}
The reduction of the {\em Match} construct has been discussed in the next section (\ref{RedMatch}).
\subsection {Reducing the Match Construct} \label{RedMatch}
Depending on whether the first pattern of the {\em pattern matching} starts with a variable, a constructor or a mixture of variables and constructors, there are three reduction rules.

\begin {enumerate}
   \item Variable Rule
   \item Constructor Rule 
   \item Mixture Rule 
\end {enumerate}
\subsubsection {Variable Rule} 
This reduction rule is used when every equation in {\em Match} starts with a {\em variable  pattern}. The reduction rule has been captured in Table \ref {varRule} below. $t_i[M/x]$ means that in term $t_i$, M has been substituted for x.        
\begin{table}[h!]
\begin{center}
\begin{tabular}{|c|c|} \hline
{\bf Before} & {\bf After} \\ 
\hline
\begin{minipage}{2in}
 \begin{alltt}

Match (u:us) 
         [
          (v\(\sb{1}\):vs\(\sb{1}\),t\(\sb{1}\)),
             .
             .
             .
          (v\(\sb{n}\):vs\(\sb{n}\),t\(\sb{n}\))
         ]
         t\(\sp{\prime}\)  

\end{alltt} 
\end {minipage} &
\begin{minipage}{2in}
\begin{alltt}

Match us 
     [
      (vs\(\sb{1}\),t\(\sb{1}\)[u/v\(\sb{1}\)]),
         .
         .
         .
      (vs\(\sb{n}\),t\(\sb{n}\)[u/v\(\sb{n}\)])
     ] 
     t\(\sp{\prime}\) 

\end{alltt} 
\end {minipage}
\tabularnewline
\hline
\end{tabular}
\caption{Reduction Rule for Variable Case}
\label{varRule}
\end{center}
\end{table}
~~\\
An example of using the {\em Variable Rule} for reducing a {\em Match} construct has been shown below in Table \ref {exvarRule}.
The {\em Match} construct in the before column in the table below is that of function $myFun$ from section \ref {ExPattMatch}.   

\begin{table}[h!]
\begin{center}
\begin{tabular}{|c|c|} \hline
{\bf Before} & {\bf After} \\ 
\hline
\begin{minipage}{3in}
 \begin{alltt}

Match [u\(\sb{1}\),u\(\sb{2}\),u\(\sb{3}\)]
  [
   ([a,Nil,ys],               P a ys),
   ([a,Cons(x,xs),Nil],       Q a xs),
   ([a,Cons(x,xs),Cons(y,ys)],R x xs y ys)
  ]
  t\(\sp{\prime}\) 

\end{alltt} 
\end {minipage} &
\begin{minipage}{3.2in}
 \begin{alltt}

Match [u\(\sb{2}\),u\(\sb{3}\)]
  [
   ([Nil,ys],               P u\(\sb{1}\) ys),
   ([Cons(x,xs),Nil],       Q u\(\sb{1}\) xs),
   ([Cons(x,xs),Cons(y,ys)],R x xs y ys)
  ]
  t\(\sp{\prime}\) 

\end{alltt} 
\end {minipage}
\tabularnewline
\hline
\end{tabular}
\caption{Example for Variable Case}
\label{exvarRule}
\end{center}
\end{table}


\subsubsection {Constructor Rule}
This reduction rule is used when every equation in {\em Match} starts with a {\em constructor pattern}. The reduction rule has been captured in Table \ref {consRule} below. 
~~\\~~\\
There can be mutiple equations in the {\em Match}
construct starting with the same constructor name. The equations with the same constructor name are grouped together by interchanging the order of equations (to which these {\em constructor patterns} belong) in the {\em Match} construct.
~~\\~~\\
It should be noted that order of two consecutive equations with constructor patterns in the beginning can be interchanged only when the constructor names in these patterns are different.
~~\\ ~~\\  
The {\em Match} construct with the grouped equations, will have the below structure where every equation $\mathbf {eq_i}$ is the list of equations starting with the same constructor $\mathbf {C_i}$. $\mathbf {++}$ is the {\em append} function.
\begin{alltt}

            Match (u:us) 
                     [
                       eq\(\sb{1}\) ++ ... ++ eq\(\sb{k}\) 
                     ]
                     t\(\sp{\prime}\)  


\end{alltt}
Consider that constructor $\mathbf {C_i}$ for equation $\mathbf {eq_i}$ takes $\mathbf {r}$ arguments, $\mathbf {C_i~b_1~\ldots~b_r}$. For succintness, instead of using $\mathbf {C_i~b_1~\ldots~b_r}$, the representation $\mathbf{C_i~bs}$ is used where $\mathbf{bs}$ represents the list of arguments that the constructor $\mathbf{C_i}$ takes.
i.e $\mathbf {~~bs = [b_1,\ldots,b_r]} .$
~~\\~~\\
Consider that $\mathbf {eq_i}$ has {\em j} equations each starting with a constructor $\mathbf {C_i}$. $\mathbf {eq_i}$ can then be represented as 
below.$\mathbf {rs}$ represents the remaining patterns of the equation,i.e the ones except the first constructor pattern.   
\begin{alltt}

           eq\(\sb{i}\) = [
                    ( ((C\(\sb{i}\) bs\(\sb{i1}\)):rs\(\sb{i1}\)),t\(\sb{i1}\)),
                          .
                          .
                          .
                    ( ((C\(\sb{i}\) bs\(\sb{ij}\)):rs\(\sb{ij}\)),t\(\sb{ij}\))
                  ]  

\end{alltt}
\begin{table}[h!]
\begin{center}
\begin{tabular}{|c|c|} \hline
{\bf Before} & {\bf After} \\ 
\hline
\begin{minipage}{3in}
\begin{alltt}

Match (u:us) 
         [
           eq\(\sb{1}\) ++ ... ++ eq\(\sb{k}\) 
         ]
         t\(\sp{\prime}\)  

\end{alltt} 
\end {minipage} &
\begin{minipage}{3in}
\begin{alltt}

    case u of 
       C\(\sb{1}\) ns\(\sb{1}\) -> Match (ns\(\sb{1}\) ++ us) eq\(\sb{1}\sp{\prime}\) t\(\sp{\prime}\) 
         .              .
         .              .
         .              .
       C\(\sb{k}\) ns\(\sb{k}\) -> Match (ns\(\sb{k}\) ++ us) eq\(\sb{k}\sp{\prime}\) t\(\sp{\prime}\) 

\end{alltt} 
\end {minipage}
\tabularnewline
\hline
\end{tabular}
\caption{Reduction Rule for Constructor Case}
\label{consRule}
\end{center}
\end{table}
~~\\
In the {\em After} column of Table \ref{consRule}, $\mathbf{ns_i}$ is the list of fresh variables of length equal to the number of arguments that the constructor $\mathbf{C_i}$ takes. Every $\mathbf{eq_i^\prime}$ has the following form.

\begin{alltt}

           eq\(\sb{i}\sp{\prime}\) = [
                    (bs\(\sb{i1}\) ++ rs\(\sb{i1}\), t\(\sb{i1}\)),
                          .
                          .
                          .
                    (bs\(\sb{ij}\) ++ rs\(\sb{ij}\), t\(\sb{ij}\))
                  ]  


\end{alltt}
An example of using the {\em Constructor Rule} for reducing a {\em Match} construct has been shown below in Table \ref {exConsRule}.
The {\em Match} construct in the before column in the table below is the {\em Match} construct from the {\em After} column of Table \ref {exvarRule} with its equations partitioned according to the constructor names in the first pattern.
~~\\ 
\begin{table}[h!]
\begin{center}
\begin{tabular}{|c|c|} \hline
{\bf Before} & {\bf After} \\ 
\hline
\begin{minipage}{3.1in}
 \begin{alltt}


Match [u\(\sb{2}\),u\(\sb{3}\)]
  (
    [([Nil,ys],P u\(\sb{1}\) ys)] ++
    [([Cons(x,xs),Nil],       Q u\(\sb{1}\) xs),
     ([Cons(x,xs),Cons(y,ys)],R x xs y ys)]
  )  
  t\(\sp{\prime}\) 


\end{alltt} 
\end {minipage} &
\begin{minipage}{3in}
 \begin{alltt}

case u\(\sb{2}\) of 
  Nil -> 
    Match [u\(\sb{3}\)] [([ys], P u\(\sb{1}\) ys)] t\(\sp{\prime}\) 

  Cons u\(\sb{4}\) u\(\sb{5}\) -> 
    Match [u\(\sb{4}\),u\(\sb{5}\),u\(\sb{3}\)]
      [
        ([x,xs,Nil],       Q u\(\sb{1}\) xs),
        ([x,xs,Cons(y,ys)],R x xs y ys)
      ]
      t\(\sp{\prime}\) 

\end{alltt} 
\end {minipage}
\tabularnewline
\hline
\end{tabular}
\caption{Example for Constructor Case}
\label{exConsRule}
\end{center}
\end{table}

\subsubsection {Mixture Rule}
This reduction rule is used when some equations in the {\em Match} construct starts with a {\em constructor pattern} and some with a {\em variable pattern}. The reduction rule has been captured in Table \ref {mixRule} below.
~~\\
\begin{table}[h!]
\begin{center}
\begin{tabular}{|c|c|} \hline
{\bf Before} & {\bf After} \\ 
\hline
\begin{minipage}{3in}
\begin{alltt}

Match us  
         [
           eq\(\sb{1}\) ++ ... ++ eq\(\sb{k}\) 
         ]
         t\(\sp{\prime}\)  

\end{alltt} 
\end {minipage} &
\begin{minipage}{3in}
\begin{alltt}

Match us eq1 
  (
    Match us eq2
      (
       ...
         (Match us eq_k t')
      )
  )

\end{alltt} 
\end {minipage}
\tabularnewline
\hline
\end{tabular}
\caption{Reduction Rule for Mixture Case}
\label{mixRule}
\end{center}
\end{table}
~~\\~~\\ 
In the {\em Before} column of Table \ref {mixRule}, the equations have been divided into {\em k} parts, $eq_1,\ldots,eq_k$. Every partition contains all the consecutive equations that begin either with a {\em variable pattern} or {\em constructor pattern}. For example - Consider the below {\em Match} construct.   
~~\\~~\\ 
\begin{alltt}
        Match [u\(\sb{1}\),u\(\sb{2}\)]
            [
             ([Nil,ys],       A ys),
             ([xs,Nil],       B xs),
             ([Cons(x,xs),ys],C x xs ys)
            ]
          t\(\sp{\prime}\) 
\end{alltt} 
~~\\ 
{\em Match} construct with partitioned equations is as below.
\begin{alltt}
        Match [u\(\sb{1}\),u\(\sb{2}\)]
          (
            [([Nil,ys],       A ys)] ++
            [([xs,Nil],       B xs)] ++
            [([Cons(x,xs),ys],C x xs ys)]
          )  
          t\(\sp{\prime}\) 
\end{alltt} 

Reduction of this {\em Match} construct using the mixture rule has been shown in the Table \ref {exMixed} below.

\begin{table}[h!]
\begin{center}
\begin{tabular}{|c|c|} \hline
{\bf Before} & {\bf After} \\ 
\hline
\begin{minipage}{3in}
 \begin{alltt}


    Match [u\(\sb{1}\),u\(\sb{2}\)]
      (
        [([Nil,ys],       A ys)] ++
        [([xs,Nil],       B xs)] ++
        [([Cons(x,xs),ys],C x xs ys)]
      )  
      t\(\sp{\prime}\) 

\end{alltt} 
\end {minipage} &
\begin{minipage}{3.2in}
\begin{alltt}

    Match [u\(\sb{1}\),u\(\sb{2}\)]
      [([Nil,ys],A ys)] 
      (
        Match [u\(\sb{1}\),u\(\sb{2}\)]
          [([xs,Nil],B xs)] 
          (
            Match [u\(\sb{1}\),u\(\sb{2}\)]
              [([Cons(x,xs),ys],C x xs ys)]
              t\(\sp{\prime}\) 
          )
      )  
      
\end{alltt} 
\end {minipage}
\tabularnewline
\hline
\end{tabular}
\caption{Example for Mixture Case}
\label{exMixed}
\end{center}
\end{table}

\subsection {Handling the Base Case} \label{BaseCase}
The base case is reached when the first argument of the {\em Match} construct which is a variable list, is empty. The base case can be divided into two cases.
\begin{enumerate}
    \item When the equation list associated with the {\em Match} construct is {\em non-empty}. Note that in this case, the pattern list associated with every equation of the equation list will be empty.
    ~~\\~~\\
    \begin{alltt}
    Match [] 
         [
          ([],t\(\sb{1}\)),
             .
             .
             .
          ([],t\(\sb{n}\))
         ] 
         t\(\sp{\prime}\) 

    \end{alltt}
    \item When the equation list associated with the {\em Match} construct is {\em empty}
    \begin{alltt}
    Match [] [] t\(\sp{\prime}\) 
    \end{alltt}
\end{enumerate}
Reduction of the base case {\em Match} constructs has been given in Table \ref{ExBaseCase} below.

\begin{table}[h!]
\begin{center}
\begin{tabular}{|c|c|} \hline
{\bf Before} & {\bf After} \\ 
\hline
\begin{minipage}{3in}
 \begin{alltt}

    Match [] 
         [
          ([],t\(\sb{1}\)),
             .
             .
             .
          ([],t\(\sb{n}\))
         ] 
         t\(\sp{\prime}\) 

\end{alltt} 
\end {minipage} &
\begin{minipage}{3.2in}
\begin{alltt}
  t\(\sb{1}\)
     
\end{alltt} 
\end {minipage}\\ 
\hline
\begin{minipage}{3in}
 \begin{alltt}

    Match [] [] t\(\sp{\prime}\) 

\end{alltt} 
\end {minipage} &
\begin{minipage}{3.2in}
\begin{alltt}
  t\(\sp{\prime}\) 
     
\end{alltt} 
\end {minipage}
\tabularnewline
\hline
\end{tabular}
\caption{Reduction Rule for Base Case}
\label{ExBaseCase}
\end{center}
\end{table}

\section {An Example of Pattern-Matching Compilation}
In this section, reduction rules described in sections \ref{RedMatch} and \ref{BaseCase} are used to  compile the {\em Match} construct for {\em myFun} function defined in section \ref{MatchCon}. The first two steps of the reductions have already been done as examples of using the {\em variable case} and {\em constructor case} in Table \ref {exvarRule} and Table \ref {exConsRule} respectively. The first {\em Match} construct in the table below is the {\em Match} construct from the {\em After} column of Table \ref {exConsRule}.

\begin{table}[h!]
\begin{center}
\begin{tabular}{|c|c|} \hline
\begin{minipage}{3.5in}
\begin{alltt}

Step 1 :-

case u\(\sb{2}\) of 
  Nil -> 
    Match [u\(\sb{3}\)] [([ys], P u\(\sb{1}\) ys)] t\(\sp{\prime}\) 
  Cons u\(\sb{4}\) u\(\sb{5}\) -> 
    Match [u\(\sb{4}\),u\(\sb{5}\),u\(\sb{3}\)]
      [
        ([x,xs,Nil],       Q u\(\sb{1}\) xs),
        ([x,xs,Cons(y,ys)],R x xs y ys)
      ]
      t\(\sp{\prime}\) 


\end{alltt} 
\end {minipage} &
\begin{minipage}{3.5in}
\begin{alltt}

Step 2 :-

case u\(\sb{2}\) of 
  Nil -> 
    Match [] [([], P u\(\sb{1}\) u\(\sb{3}\) )]
  Cons u\(\sb{4}\) u\(\sb{5}\) -> 
    Match [u\(\sb{5}\),u\(\sb{3}\)]
      [
        ([xs,Nil],       Q u\(\sb{1}\) xs),
        ([xs,Cons(y,ys)],R u\(\sb{4}\) xs y ys)
      ]
      t\(\sp{\prime}\) 

\end{alltt} 
\end {minipage}\\ 
\hline 
\begin{minipage}{3.5in}
\begin{alltt}

Step 3 :-

case u\(\sb{2}\) of 
  Nil -> 
    P u\(\sb{1}\) u\(\sb{3}\) 
  Cons u\(\sb{4}\),u\(\sb{5}\) -> 
    Match [u\(\sb{3}\)]
      (
       [([Nil],       Q u\(\sb{1}\) u\(\sb{5}\))] ++
       [([Cons(y,ys)],R u\(\sb{4}\) u\(\sb{5}\) y ys)]
      )
      t\(\sp{\prime}\)

\end{alltt} 
\end {minipage} &
\begin{minipage}{3.5in}
\begin{alltt}

Step 4 :-

case u\(\sb{2}\) of 
  Nil -> P u\(\sb{1}\) u\(\sb{3}\)
  Cons u\(\sb{4}\),u\(\sb{5}\) -> 
    case u3 of 
      Nil -> Q u\(\sb{1}\) u\(\sb{5}\)
      Cons u\(\sb{6}\) u\(\sb{7}\) -> 
        Match [u\(\sb{6}\), u\(\sb{7}\)]
              ([y,ys],R u\(\sb{4}\) u\(\sb{5}\) y ys)
              t' 


\end{alltt} 
\end {minipage} \\ 
\hline
\begin{minipage}{3.5in}
\begin{alltt}

Step 5 :-

case u\(\sb{2}\) of 
  Nil -> P u\(\sb{1}\) u\(\sb{3}\)
  Cons u\(\sb{4}\),u\(\sb{5}\) -> 
    case u3 of 
      Nil -> Q u\(\sb{1}\) u\(\sb{5}\)
      Cons u\(\sb{6}\) u\(\sb{7}\) -> 
        Match []
              ([],R u\(\sb{4}\) u\(\sb{5}\) u\(\sb{6}\) u\(\sb{7}\))
              t' 

\end{alltt} 
\end {minipage} &
\begin{minipage}{3.5in}
\begin{alltt}
Step 6 :-

case u\(\sb{2}\) of 
  Nil -> P u\(\sb{1}\) u\(\sb{3}\)
  Cons u\(\sb{4}\) u\(\sb{5}\) -> 
    case u3 of 
      Nil -> Q u\(\sb{1}\) u\(\sb{5}\)
      Cons u\(\sb{6}\) u\(\sb{7}\) -> R u\(\sb{4}\) u\(\sb{5}\) u\(\sb{6}\) u\(\sb{7}\)



\end{alltt} 
\end {minipage}
\tabularnewline
\hline
\end{tabular}
\caption{Reduction of Match for myFun}
\label{exMyFun}
\end{center}
\end{table}
~~\\~~\\
The function definition of {\em myFun} with compiled {\em pattern-matching} is as below :- 
\newpage
\begin{alltt}
fun myFun =
   u\(\sb{1}\),u\(\sb{2}\),u\(\sb{3}\) ->
        case u\(\sb{2}\) of 
          Nil -> P u\(\sb{1}\) u\(\sb{3}\)
          Cons u\(\sb{4}\) u\(\sb{5}\) -> 
            case u3 of 
              Nil -> Q u\(\sb{1}\) u\(\sb{5}\)
              Cons u\(\sb{6}\) u\(\sb{7}\) -> R u\(\sb{4}\) u\(\sb{5}\) u\(\sb{6}\) u\(\sb{7}\)



\end{alltt} 

\section {Totality of MPL Functions}
Currently every MPL function is required to be total for successful {\em pattern-matching} compilation. Functions that are not total will throw an error in the compilation owing to the missing constructors.

\end {document}