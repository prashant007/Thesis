\documentclass[11pt]{article}

\title{Lambda Lifting Transformation}
\usepackage[utf8]{inputenc}
\usepackage{tikz}
\usepackage{tikz-3dplot}
\usepackage{tikz-cd}
\usepackage{xcolor}
\usepackage{proof}
\usepackage {alltt}

\usepackage{graphicx} % Allows including images
\usepackage{booktabs} % Allows the use of \toprule, \midrule and \bottomrule in tables
\usepackage{listings}
\usepackage[export]{adjustbox}
\usepackage{verbatim}
\tikzset
  {cross/.style={cross out, draw=black, minimum size=2*(#1-\pgflinewidth), inner sep=0pt, outer sep=0pt},cross/.default={1pt}}


\usepackage{amsmath }
\usepackage{framed}
\usepackage{proof}
\usepackage{enumerate,xspace,stmaryrd}
\usepackage{amsmath,amssymb,latexsym}
\usepackage{textcomp,xspace}
\usepackage[top=2cm, bottom=2.5cm, right=2.5cm, left=2.5cm]{geometry}\usepackage{latexsym}  %For plain TeX symbols, such as \Box used in \qed
\usepackage{euscript}  %%Euler Script font
\usepackage {mdframed}
\usepackage{ifpdf}
\usepackage {alltt}
\renewcommand{\ttdefault}{txtt}
\ifpdf
  \usepackage{epstopdf}
\fi 
\usepackage{etoolbox}
\BeforeBeginEnvironment{tabular}{\begin{center}\small}
\AfterEndEnvironment{tabular}{\end{center}}

\tikzstyle{stage} = [rectangle,minimum width=2.8cm,rounded corners,
                     minimum height=1.8cm,text centered, draw=black]
\tikzstyle{arrow} = [very thick,->,>=stealth]

\begin{document}

\maketitle
\section {Sequential MPL Constructs} 
Sequential MPL constructs can be divided into three categories:
\begin{itemize}
  \item {\bf Data Type Constructs:} These constructs are used with data types. They are {\sf case}, {\sf constructor} and {\sf fold} constructs.
  \item {\bf Codata Type Constructs:}
  \item {\bf Non Data/Codata Constructs:} These are general MPL constructs that are not used with either data or codata types. They are {\sf function call}, {\sf where}, {\sf variable}, {\sf constant} and {\sf if} constructs.
\end{itemize}
In this following sections non data/codata constructs are discussed first, followed by data type constructs and codata type constructs.
\section {Non Data/Codata Constructs}
These are general MPL constructs that are not used with either data or codata types. They are {\sf function call}, {\sf where}, {\sf variable}, {\sf constant} and {\sf if} constructs.
\subsection {function call construct}
\subsection {if construct}
\subsection {variable construct}
\subsection {constant construct}
\subsection {where construct}

\section {Data Type Constructs}
These constructs are used with data types. They are {\sf case}, {\sf constructor} and {\sf fold} constructs.
\subsection {case construct}
\subsection {data constructor construct}
\subsection {fold construct}

\section {Codata Type Constructs}
 These constructs are used with codata types. They are {\sf record} and {\sf destructor} constructs.
\subsection {record construct}
\subsection {destructor construct}
\subsection {product construct}
\end {document}